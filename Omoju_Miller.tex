%%% A template to produce a nice-looking Curriculum Vitae.
%%% Adapted from Kieran Healy 
%%% Most recent version is at http://kjhealy.github.com/kjh-vita
%%%
%%% ------------------------------------------------------------------------
%%% Requirements that are included in a modern tex distribution:
%%% ------------------------------------------------------------------------
%%% xelatex
%%% fontspec.sty
%%% hyperrref.sty
%%% xunicode.sty
%%% color.sty
%%% url.sty
%%% fancyhdr.sty
%%% memoir.cls
%%% fontawesome.sty
%%%
%%% 
%%% 
%%% ------------------------------------------------------------------------
%%% Requirements from https://github.com/kjhealy/latex-custom-kjh
%%% ------------------------------------------------------------------------
%%% org-preamble-xelatex.sty
%%% memoir-article-styles.sty
%%%
%%% ------------------------------------------------------------------------
%%% Optional
%%% ------------------------------------------------------------------------
%%% git
%%% vc.sty
%%% revnum.sty
%%% Fonts
%%%
%%% ------------------------------------------------------------------------
%%% Note
%%%------------------------------------------------------------------------
%%% Because this is a hand-tweaked file, be on the look out for \medksip, 
%%% \bigskip and \newpage commands here and there, which are used to balance
%%% the layout or avoid widows & orphans, etc. You should of course add or 
%%% remove these as needed.
%%%------------------------------------------------------------------------

\documentclass[11pt,article,oneside]{memoir}  
\usepackage{geometry}
 \geometry{
 a4paper,
 total={170mm,257mm},
 left=40mm,
 right=10mm,
 top=5mm,
 bottom=20mm,
 }
\usepackage{org-preamble-xelatex} 
\usepackage{fontawesome,url}
\usepackage{enumitem} %get rid of spaces in listed

%%%------------------------------------------------------------------------
%%% Metadata
%%%------------------------------------------------------------------------

%% Change as needed. Or just add me as a coauthor. Only some of these are 
%% used below in the hyperref declaration and address banner section.
\def\myauthor{Omoju Thomas Miller}
\def\mytitle{Vita}
\def\mycopyright{\myauthor}
\def\mykeywords{}
\def\mybibliostyle{plain}
\def\mybibliocommand{}
\def\mysubtitle{}
\def\myemail{omojumiller@berkeley.edu}
\def\mytwitter{@omojumiller}
\def\mylinkedin{http://buff.ly/1TXyVq5}
\def\mygithub{https://github.com/omoju}
\def\myversion{}
\def\myrevision{}


%% \def\myaffiliation{Duke University}
\def\myauthor{Omoju Thomas Miller}
\date{} % not used (revision control instead)
\def\mykeywords{Omoju, Miller, Omoju Miller, Vita, CV, Resume, Data Scientist}

%%%------------------------------------------------------------------------  
%%% Git version tracking 
%%%------------------------------------------------------------------------

%% If you don't use git or the vc package (from CTAN), comment this out.
%% If you comment it out, be sure to remove the \rfoot comment below, too.
%% \input{vc}

%%%------------------------------------------------------------------------
%%% Document
%%%------------------------------------------------------------------------
\begin{document}

%% Choose fonts for use with xelatex
%% Minion and Myriad are widely available, from Adobe. 
%% Pragmata is available to buy at http://www.fsd.it/fonts/pragma.htm
%% and is worth every penny. Any good monospace font will work fine, though.
%% Consolas or inconsolata are good alternatives.
\setromanfont[Mapping={tex-text}, 
	Numbers={OldStyle},
	Ligatures={Common}]{Minion Pro}
\setsansfont[Mapping=tex-text,
	Ligatures={Common}, 
	Colour=AA0000]{Myriad Pro}
\setmonofont[Mapping=tex-text,Scale=0.9]{Consolas} 

\newfontface\scheader[SmallCapsFont={Minion Pro},SmallCapsFeatures={Letters=SmallCaps}]{Minion Pro}

\newfontface\addressblock[Mapping={tex-text}, 
	Numbers={OldStyle},
	Ligatures={Common}]{Minion Pro Medium}


%%%------------------------------------------------------------------------
%%% Local commands
%%%------------------------------------------------------------------------

%% Marginal header
%% Note: as the document goes on you may need to introduce a (gradually increasing)
%% \vspace element to keep the marginal header pleasingly aligned with the first 
%% item in the body text. Like this: \marginhead{{\vskip 0.4em}Grants}, or 
%% \marginhead{{\vskip 0.8em}Service}. Experiment as needed.
\newcommand{\marginhead}[1]{\marginpar{\textsf{{\footnotesize\vspace{-1em}\flushright #1}}}}


%% [optional] custom ampersand (font consistent with the one chosen above)
% \newcommand{\amper}{{\fontspec[Scale=.95,Colour=AA0000]{Minion Pro Medium}\selectfont\&\,}}

%% No bullets on labels
\renewcommand{\labelitemi}{~} 

%% Custom hanging indent for vita items
\def\ind{\hangindent=1 true cm\hangafter=1 \noindent}
%\def\ind{\hangindent=18pt\hangafter=1 \noindent}
\def\labelitemi{~}
\renewcommand{\labelitemii}{~}

%%%------------------------------------------------------------------------
%%% Page layout
%%%------------------------------------------------------------------------

% These lines will insert git revision info in the footer, using the vc
% package---see docs for vc package for details. Comment out this line
% if you're not using vc, and also remove the \input{vc} line above.
%\pagestyle{kjh}
\thispagestyle{kjhgit}


%%%------------------------------------------------------------------------
%%% Address and contact block
%%%------------------------------------------------------------------------
\begin{minipage}[t]{2.95in}
  
\end{minipage}
\hfill     
%\begin{minipage}[t]{0.0in}
% dummy (needed here)
%\end{minipage}
\hfill
\begin{minipage}[t]{2.6in}
  \flushright \footnotesize  \addressblock 
  {\scriptsize  \texttt{\href{http://twitter.com/omojumiller}{\mytwitter}} \, \faTwitter }  \\ 
  {\scriptsize  \texttt{\href{mailto:\myemail}{\myemail}} \, \faEnvelope} \\
  {\scriptsize  \texttt{\href{\mylinkedin}{\mylinkedin}} \, \faLinkedin} \\
  {\scriptsize  \texttt{\href{\mygithub}{\mygithub}} \, \faGithub} 
\end{minipage}

\medskip

%% Name 
\noindent{\HUGE\scheader \textsc{\href{http://omojumiller.com/portfolio/}{omoju thomas miller}}}

\bigskip       
\noindent{\large\scheader \textsc{Ex-Google | White House Influencer | UC Berkeley Ph.D.}}
\reversemarginpar

%% Skills
\marginhead{\sffamily {{\vskip -0.23em} skills}}
\medskip

\ind \textbf{Programming Languages}: Python, Prolog, C, C++, Ruby(basic), SQL

\ind \textbf{Frameworks}: XGBoost, NLTK, Pandas, Numpy, Scikit-Learn, Word2Vec

\ind \textbf{Version Control Systems}: Git
\bigskip 

%% Projects
\marginhead{\sffamily {\vskip 0.35em} projects}
\medskip

\ind \href{https://github.com/omoju/creatingCustomerSegments}{Creating Customer Segments} \hfill Sept 2016 
\begin{itemize}[noitemsep,nolistsep]
\item[-] Built an unsupervised learning model to identify customer segments from data.

\end{itemize}

\ind \href{https://github.com/omoju/SWEXSYS}{Semantic Web Expert System Shell} (SWEXSYS) \hfill Sept 2006 
\begin{itemize}[noitemsep,nolistsep]
\item[-] Created an intelligent system platform capable of reasoning with multiple ontologies.
\end{itemize} 

\bigskip


%% Professional Experience
\marginhead{\sffamily {\vskip 0.35em} professional \newline experience}
\medskip

\ind {\large \textbf{Insight Datascience} - \textit{Data Science Fellow}} \hfill Palo Alto, CA | Jan 2017-Present
\begin{itemize}[nolistsep]
\item[-]Built a large-scale hierarchical classification model for a YC backed management service company.
\item[-]Scaled solution to accommodate growing taxonomy of item categories.
\end{itemize} 

\ind {\large \textbf{UC Berkeley} - \textit{CSEd Doctoral Candidate}} \hfill Berkeley, CA | Dec 2013-May 2016
\begin{itemize}[nolistsep]
\item[-] Applied machine learning to investigate student decision making around computer science choices.
\item[-] Developed a data science unit within a historical \href{http://www.whitehouse.gov/the-press-office/2014/12/08/fact-sheet-new-commitments-support-computer-science-education}{intro CS course}, CS10.
\end{itemize}  

\ind {\large \textbf{Google} - \textit{Technology Portfolio Manager}} \hfill Mountain View, CA | Feb 2014-Dec 2015
\begin{itemize}[nolistsep]
\item[-]Co-led a \$15M CSEd portfolio. Grew the portfolio by 25\%.
\item[-]Co-led the development of vocabulary glossary for Android Development for Beginners course.
\end{itemize} 

\ind {\large \textbf{ICSI\footnote{International Computer Science Institute}} - \textit{Graduate Researcher}} \hfill Berkeley, CA | Mar 2011-Dec 2012
\begin{itemize}[nolistsep]
\item[-] Built a conceptual metaphor \href{http://metaphor.icsi.berkeley.edu}{Semantic MediaWiki} capable of ingesting data in English, and Farsi.
\item[-] Contributed to the design of conceptual metaphor framework.
\end{itemize} 

\ind {\large \textbf{UniversityNow} - \textit{Data Scientist (Intern)}} \hfill San Francisco, CA | Jun 2012 -Aug 2012
\begin{itemize}[nolistsep]
\item[-]Led a 3 person team responsible for creating an analytics console.
\item[-]Pulled and integrated data from two separate database management systems.
\end{itemize} 

\ind {\large \textbf{Center for Advanced Sensors} - \textit{Graduate Researcher}} \hfill Memphis, TN | Aug 2003-Nov 2006
\begin{itemize}[nolistsep]
\item[-]Co-designed \href{https://scholar.google.com/scholar?hl=en&q=Building+a+Sensor+Ontology\%3A+A+Practical+Approach+Leveraging+ISO+and+OGC+Models.&btnG=&as_sdt=1\%2C43&as_sdtp=}{OntoSensor}, a sensor ontology using Semantic Web Ontology Language (OWL).
\item[-]Built an intelligent reasoning agent capable of reasoning with data in XML and RDF formats
\end{itemize}

\ind {\large \textbf{Thoughtware Learning Technologies} - \textit{Software Engr}} \hfill Memphis, TN | Aug 1999-Nov 2001
\begin{itemize}[nolistsep]
\item[-]Contributed to the development of eLearning workforce development management system.
\item[-]Quality assurance for an eLearning management system.
\end{itemize} 

\bigskip 

%% Teaching Experience
\marginhead{\sffamily {\vskip 0.35em} teaching \newline experience}
\medskip

\ind {\large \textbf{UC Berkeley} - \textit{Graduate Student Instructor}} \hfill Berkeley, CA | Spring 2011
\begin{itemize}[noitemsep,nolistsep] 
\item[-] Introduction to C++ programming
\end{itemize} 

\ind {\large \textbf{University of Memphis} - \textit{Adjunct Faculty}} \hfill Memphis, TN | Fall 2009, Spring 2010
\begin{itemize}[noitemsep,nolistsep]
\item[-] C programming for Engineers
\end{itemize}

\bigskip 

%% Education
\marginhead{\sffamily {{\vskip -0.23em} education}}

\ind \textbf{Udacity} - Machine Learning Nanodegree  \hfill (Expected 2017)

\ind \textbf{UC Berkeley} - Computer Science Education, Ph.D. \hfill 2016

\ind \textbf{University of Memphis} - Intelligent Systems (EECE), M.S. \hfill 2004

\ind \textbf{University of Memphis} - Computer Science, B.S. \hfill 2001

\bigskip 

%% Honors
\marginhead{\sffamily {\vskip 0.6em}grants,\newline honors, \newline \& awards}
\medskip

\ind Member, World Economic Forum (Expert Network: AI and Robotics), Davos, Switzerland \hfill 2017

\ind Winner, White House Dept of Edu EdDataPalooza, Washington DC \hfill 2014

\ind Bloomberg Beta Future Founder, Bloomberg Beta, San Francisco CA \hfill 2014

\end{document}
