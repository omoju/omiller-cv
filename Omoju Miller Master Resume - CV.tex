
%%% A template to produce a nice-looking Curriculum Vitae.
%%% Adapted from Kieran Healy 
%%% Most recent version is at http://kjhealy.github.com/kjh-vita
%%%
%%% ------------------------------------------------------------------------
%%% Requirements that are included in a modern tex distribution:
%%% ------------------------------------------------------------------------
%%% xelatex
%%% fontspec.sty
%%% hyperrref.sty
%%% xunicode.sty
%%% color.sty
%%% url.sty
%%% fancyhdr.sty
%%% memoir.cls
%%% fontawesome.sty
%%%
%%% 
%%% 
%%% ------------------------------------------------------------------------
%%% Requirements from https://github.com/kjhealy/latex-custom-kjh
%%% ------------------------------------------------------------------------
%%% org-preamble-xelatex.sty
%%% memoir-article-styles.sty
%%%
%%% ------------------------------------------------------------------------
%%% Optional
%%% ------------------------------------------------------------------------
%%% git
%%% vc.sty
%%% revnum.sty
%%% Fonts
%%%
%%% ------------------------------------------------------------------------
%%% Note
%%%------------------------------------------------------------------------
%%% Because this is a hand-tweaked file, be on the look out for \medksip, 
%%% \bigskip and \newpage commands here and there, which are used to balance
%%% the layout or avoid widows & orphans, etc. You should of course add or 
%%% remove these as needed.
%%%------------------------------------------------------------------------

\documentclass[11pt,article,oneside]{memoir}   
\usepackage{org-preamble-xelatex} 
\usepackage{fontawesome,url}
\usepackage{enumitem} %get rid of spaces in listed


%%%------------------------------------------------------------------------
%%% Metadata
%%%------------------------------------------------------------------------

%% Change as needed. Or just add me as a coauthor. Only some of these are 
%% used below in the hyperref declaration and address banner section.
\def\myauthor{Omoju Thomas Miller}
\def\mytitle{Vita}
\def\mycopyright{\myauthor}
\def\mykeywords{}
\def\mybibliostyle{plain}
\def\mybibliocommand{}
\def\mysubtitle{}
\def\myemail{omojumiller@gmail.com}
\def\myweb{https://omojumiller.com}
\def\mytwitter{@omojumiller}
\def\mylinkedin{http://buff.ly/1TXyVq5}
\def\mygithub{https://github.com/omoju}
\def\myversion{}
\def\myrevision{}


%% \def\myaffiliation{Duke University}
\def\myauthor{Omoju Thomas Miller}
\date{} % not used (revision control instead)
\def\mykeywords{Omoju, Thomas Miller, Omoju Thomas Miller, Vita, CV, Resume, Data Scientist}

%%%------------------------------------------------------------------------  
%%% Git version tracking 
%%%------------------------------------------------------------------------

%% If you don't use git or the vc package (from CTAN), comment this out.
%% If you comment it out, be sure to remove the \rfoot comment below, too.
%% \input{vc}

%%%------------------------------------------------------------------------
%%% Document
%%%------------------------------------------------------------------------
\begin{document}

%% Choose fonts for use with xelatex
%% Minion and Myriad are widely available, from Adobe. 
%% Pragmata is available to buy at http://www.fsd.it/fonts/pragma.htm
%% and is worth every penny. Any good monospace font will work fine, though.
%% Consolas or inconsolata are good alternatives.
\setromanfont[Mapping={tex-text}, 
	Numbers={OldStyle},
	Ligatures={Common}]{Minion Pro}
\setsansfont[Mapping=tex-text,
	Ligatures={Common}, 
	Colour=AA0000]{Myriad Pro}
\setmonofont[Mapping=tex-text,Scale=0.72]{Consolas} 

\newfontface\scheader[SmallCapsFont={Minion Pro},SmallCapsFeatures={Letters=SmallCaps}]{Minion Pro}

\newfontface\addressblock[Mapping={tex-text}, 
	Numbers={OldStyle},
	Ligatures={Common}]{Minion Pro Medium}


%%%------------------------------------------------------------------------
%%% Local commands
%%%------------------------------------------------------------------------

%% Marginal header
%% Note: as the document goes on you may need to introduce a (gradually increasing)
%% \vspace element to keep the marginal header pleasingly aligned with the first 
%% item in the body text. Like this: \marginhead{{\vskip 0.4em}Grants}, or 
%% \marginhead{{\vskip 0.8em}Service}. Experiment as needed.
\newcommand{\marginhead}[1]{\marginpar{\textsf{{\footnotesize\vspace{-1em}\flushright #1}}}}


%% [optional] custom ampersand (font consistent with the one chosen above)
% \newcommand{\amper}{{\fontspec[Scale=.95,Colour=AA0000]{Minion Pro Medium}\selectfont\&\,}}

%% No bullets on labels
\renewcommand{\labelitemi}{~} 

%% Custom hanging indent for vita items
\def\ind{\hangindent=1 true cm\hangafter=1 \noindent}
%\def\ind{\hangindent=18pt\hangafter=1 \noindent}
\def\labelitemi{~}
\renewcommand{\labelitemii}{~}

%%%------------------------------------------------------------------------
%%% Page layout
%%%------------------------------------------------------------------------

% These lines will insert git revision info in the footer, using the vc
% package---see docs for vc package for details. Comment out this line
% if you're not using vc, and also remove the \input{vc} line above.
\pagestyle{kjh}
\thispagestyle{kjhgit}


%%%------------------------------------------------------------------------
%%% Address and contact block
%%%------------------------------------------------------------------------
\begin{minipage}[t]{2.95in}
  
\end{minipage}
\hfill     
%\begin{minipage}[t]{0.0in}
% dummy (needed here)
%\end{minipage}
\hfill
\begin{minipage}[t]{1.3in}
  \flushright \footnotesize  \addressblock 
  {\scriptsize  \texttt{\href{http://twitter.com/omojumiller}{\mytwitter}} \, \faTwitter }  \\ 
  {\scriptsize  \texttt{\href{mailto:\myemail}{\myemail}} \, \faEnvelope} \\
  {\scriptsize  \texttt{\href{\mylinkedin}{\mylinkedin}} \, \faLinkedin} \\
  {\scriptsize  \texttt{\href{\mygithub}{\mygithub}} \, \faGithub} \\
  {\scriptsize  \texttt{\href{\myweb}{\myweb}} \, \faGlobe}
\end{minipage}

\medskip

%% Name 
\noindent{\LARGE\scheader \textsc{omoju thomas miller}}
\reversemarginpar

\bigskip       

%% Skills
\marginhead{\sffamily {{\vskip -0.23em} skills}}
\medskip

\ind Programming Languages: Python, C, C++, Ruby, HTML, Prolog

\ind Version Control Systems: Git

\bigskip 

%% Projects
\marginhead{\sffamily {\vskip 0.35em} projects}
\medskip

\ind \href{https://github.com/omoju/studentIntervention}{Identify at risk students}
\begin{itemize}[noitemsep,nolistsep]
\item[-] Created supervised learning model to classify students needing learning intervention. Implemented using Jupyter notebooks, with Scikit-learn, Pandas and Numpy python packages.
\item[-] Implemented three supervised learning models: Decision Trees, Random Forest, and SVM.
\item[-] Tuned SVM model to improve classification accuracy by 11.2\% over Decision Tree bench mark.
\end{itemize} 

\ind \href{https://github.com/omoju/hiphopathy}{Hiphopathy}
\begin{itemize}[noitemsep,nolistsep]
\item[-] Designed and created a data science unit using rap lyrics for the \href{http://bjc.berkeley.edu/}{BJC}-CS10 curriculum. Implemented curriculum using Jupyter notebook and Python Natural Language Tool Kit (NLTK) package.
\item[-] Project was implemented on edX MOOClet, \textit{BJC.3x: Data, Information and the Internet}. Course launched over 20,000 students.
\item[-] Curricula unit had a significant positive effect on female students' belief about their ability to do well in CS classes and their ability to program.
\end{itemize} 

\ind \href{https://github.com/omoju/DissertationDataAnalysis}{Model Dynamics of Gender in Intro CS}
\begin{itemize}[noitemsep,nolistsep]
\item[-] Created a supervised learning model to understand the dynamics of gender in intro CS at Berkeley. Implemented using Jupyter notebooks, with Scikit-learn, Pandas and Numpy python packages.
\item[-] Implemented a Random Forest Classifier to identify important features that play a role in the experience of intro CS at UC Berkeley.
\item[-] Deployed a SVM learner \textit{biasing} the $C$ soft-margin hyper-parameter to correct for label class imbalance.
\end{itemize} 

\ind \href{https://github.com/omoju/investigatingWhyURMsChooseCS}{Investigate why Historically Underrepresented Students Choose CS}
\begin{itemize}[noitemsep,nolistsep]
\item[-] Designed comprehensive data driven investigation using statistical models to investigate why URMs choose to advance in the CS Major. Implemented using Jupyter notebooks, with Matplotlib, Pandas and Numpy python packages.
\item[-] Created a validated survey instrument to explore the role of computational thinking, mentorships, CS belonging, self-reported ability and pre-collegiate preparation on the choice to advance in undergraduate CS major.  
\end{itemize} 

\ind \href{https://developers.google.com/android/for-all/vocab-words/}{Vocabulary Glossary} - Android Development for Beginners course
\begin{itemize}[noitemsep,nolistsep]
\item[-] Created explicit mental-models to communicate computation ideas via verbal and nonverbal materials.
\item[-] Co-Built Android For All Vocabulary Glossary for Android Development for Beginners course in partnership with Udacity. Course launched over 50,000 students.
\end{itemize} 

\ind \href{https://www.youtube.com/channel/UCZK66JujoN3KY3sak2kEa2w?&ab_channel=ThisIsGRIT}{\#ThisIsGrit}
\begin{itemize}[noitemsep,nolistsep]
\item[-] Created \#ThisIsGrit social media campaign as a response to under-matching problem highlight by the department of education.
\item[-] Created several videos---role models in society speak of a time when things in education were really tough and they stuck it out---of high-achieving historically underrepresented persons who shared their \#ThisIsGrit stories.
\item[-] Inspired first lady \textbf{Michelle Obama}'s \href{https://www.whitehouse.gov/reach-higher}{Reach Higher} program. 
\end{itemize} 

\ind \href{https://github.com/omoju/SWEXSYS}{Semantic Web Expert System Shell} (SWEXSYS)
\begin{itemize}[noitemsep,nolistsep]
\item[-] Created an intelligent system platform capable of reasoning from multiple ontologies using Resource Description Framework (RDF), Rule Markup Language (RuleML), as well as other knowledge expressed as functional, structural, or causal models.
\item[-] Built system using Prolog and XPCE package.
\item[-] Adapted SWEXSYS platform to sensor fusion using Dempster-Shafer theory of uncertainty.
\end{itemize} 

\bigskip 

  
%% Professional Experience
\marginhead{\sffamily {\vskip 0.35em} profession \newline experience}
\medskip

\noindent\emph{Experience \vspace{0.05in}}


\ind Learners Guild, Chair Advisory Board, Oakland, CA, December 2015 - Present.
\begin{itemize}[noitemsep,nolistsep]
\item[-]Built inaugural advisory board.
\item[-]Worked with CEO to identify areas where support can add a significant performance difference to the company's bottom line.
\end{itemize} 

\ind Google, Technology Portfolio Manager, San Francisco, CA, February 2014 - December 2015. 
\begin{itemize}[noitemsep,nolistsep]
\item[-]Co-led a \$15M CSEd portfolio. Grew the portfolio by 25\%.
\item[-]Help launch CS in Media initiative aimed at increasing Computer Science awareness in media.
\item[-]Help launch Made With Code, a google program aimed at increasing the participation of girls in computing.
\item[-]Co-led the development of vocabulary glossary for Android Development for Beginners course
\end{itemize} 

\ind UC Berkeley, CSEd Doctoral Candidate, Berkeley, CA, December 2013 - December 2015.
\begin{itemize}[noitemsep,nolistsep]
\item[-] Built innovative data science unit within a historically \href{http://www.whitehouse.gov/the-press-office/2014/12/08/fact-sheet-new-commitments-support-computer-science-education}{ground breaking} intro CS course, CS10.
\item[-] Designed comprehensive data driven investigation using statistical models and machine learning to better understand the dynamics of gender in intro CS at Berkeley.
\item[-]Responsible for introduction of python programming language as part of official BJC curriculum.
\end{itemize} 

\ind International Computer Science Institute, Graduate Researcher, Berkeley, CA, March 2011 - December 2012.
\begin{itemize}[noitemsep,nolistsep]
\item[-] Co-designed and implemented a \href{http://metaphor.icsi.berkeley.edu}{Conceptual Metaphor Semantic MediaWiki System} to capture and model conceptual metaphors in English, Spanish, Farsi and Russian. 
\item[-] Contributed to the design of conceptual metaphor framework as part of AI group working on computational linguistics and reasoning with particular emphasis on metaphor cognition.
\end{itemize} 

\ind UniversityNow, Semantic Data Scientist Intern, San Francisco, CA June 2012 - August 2012.
\begin{itemize}[noitemsep,nolistsep]
\item[-]Led a team to design and create a data analytics console to improve business intelligence and illuminate how students are progressing on the learning platform.
\end{itemize} 

\ind TEDxEuclidAve, Founder and Curator, Berkeley, CA August 2011 - August 2012.
\begin{itemize}[noitemsep,nolistsep]
\item[-]Led the planning of \href{http://www.ted.com/tedx/events/3790}{TEDx}EuclidAve 100+ member conference, bringing together innovative thinkers across technology, design and entertainment in service of social entrepreneurship.
\end{itemize} 

\ind Thoughtware Learning Technologies, Inc., Software Engineer, Memphis, TN August 1999 - November 2001.
\begin{itemize}[noitemsep,nolistsep]
\item[-]Contributed to the development of Internet based workforce development management system.
\end{itemize} 

\bigskip 


%% Education
\marginhead{\sffamily {{\vskip -0.23em} education}}

\ind Nanodegree, Machine Learning, Udacity, 2016.\vspace{0.05in}

\ind Ph.D, Computer Science Education, UC Berkeley, 2015.

\ind \hspace{0.35in} \footnotesize Dissertation: \emph{HipHopathy, A Socio-Curricular Study of Introductory Computer Science.} \normalsize \vspace{0.05in}

\ind M.S., Electrical and Computer Engineering, University of Memphis, 2004. 

\ind \hspace{0.35in} \footnotesize Thesis: \emph{Semantic Web Expert System (SWEXSYS) Shell.} \normalsize \vspace{0.01in}

\ind B.S., Computer Science, University of Memphis, 2001.
\bigskip 


%% Publications
\marginhead{\sffamily {\vskip 0.35em} publications}
\medskip

\noindent\emph{Conference proceedings \vspace{0.05in}}
 
%% Use revnumerate environment if numbered publications are needed. 
%% (Include it above in the preamble).
%% \renewcommand{\labelenumi}{\textsc{a}\theenumi.}
%% \begin{revnumerate}

\ind Forthcoming. Omoju Miller. ``Gaining Insights into the Effects of Culturally Responsive Curriculum on Historically Underrepresented Students' Desire for Computer Science.'' \emph{Proceedings of the 123rd ASEE Annual Conference, New Orleans, LA.}

\ind D.J. Russomanno, C. Kothari and O. Thomas. 2005. ``\href{https://scholar.google.com/citations?view_op=view_citation&hl=en&user=E7z_wrwAAAAJ&sortby=pubdate&citation_for_view=E7z_wrwAAAAJ:u5HHmVD_uO8C}{Building a Sensor Ontology: A Practical Approach Leveraging ISO and OGC Models}.'' \emph{The 2005 International Conference on Artificial Intelligence, Las Vegas, NV} 637--643. 

\ind D.J. Russomanno, C. Kothari and O. Thomas. 2005. ``\href{http://scholar.google.com/citations?view_op=view_citation&hl=en&user=E7z_wrwAAAAJ&citation_for_view=E7z_wrwAAAAJ:u-x6o8ySG0sC}{Sensor Ontologies: From Shallow to Deep Models.}'' \emph{Proceedings of the 37th Southeastern Symposium on Systems Theory, IEEE Press, Tuskegee, AL} 107--112. 

\ind O. Thomas and D.J. Russomanno. 2005. ``\href{http://scholar.google.com/citations?view_op=view_citation&hl=en&user=E7z_wrwAAAAJ&citation_for_view=E7z_wrwAAAAJ:d1gkVwhDpl0C}{Applying the Semantic Web Expert System Shell to Sensor Fusion using Dempster-Shafer Theory.}'' \emph{Proceedings of the 37th Southeastern Symposium on Systems Theory, IEEE Press, Tuskegee, AL}. 11--14.

\ind O.A. Thomas and D.J. Russomanno. 2004. ``\href{http://scholar.google.com/citations?view_op=view_citation&hl=en&user=E7z_wrwAAAAJ&citation_for_view=E7z_wrwAAAAJ:2osOgNQ5qMEC}{SWEXSYS: A Semantic Web Expert System Shell}.'' \emph{Proceedings of the IASTED International Conference on Artificial Intelligence and Applications, part of the 23rd Multi-Conference on Applied Informatics, Innsbruck, Austria} 240--245. 

\bigskip

\noindent\emph{Workshops and Presentations  \vspace{0.05in}}

%\renewcommand{\labelenumi}{\textsc{r}\theenumi.}
%\begin{revnumerate}

\ind Jennifer Arguello, Tiffany Price, Alexis Martin, Frieda McAlear, Omoju Miller, Dan Garcia. 2015. ``Engaging Underrepresented Youth in Computer Science."  \emph{Computer Science Teachers Association (CSTA) in Grapevine, TX.}

\ind Omoju Miller. 2015. ``Towards a Global Framework for Culturally Resonant Learning.'' \emph{2015 Sage Assembly in Paris, France.} 

\ind Omoju Miller and Oluwole Solana. 2014. ``From San Francisco to Johannesburg, introducing girls to Computer Science in Industry supported programs.'' \emph{2014 Grace Hopper Conference in Pheonix, AZ.}

\ind Daniel D Garcia, Brian Harvey, Tiffany Barnes, Dan Armendariz, Jon McKinsey, Zachary MacHardy, Omoju Miller, Barry Peddycord III, Eugene Lemon, Sean Morris, Josh Paley. 2014. ``\href{http://dl.acm.org/citation.cfm?id=2539026}{AP CS Principles and the Beauty and Joy of Computing Curriculum.'' }\emph{Proceedings of the 45th ACM Technical Symposium on Computer Science Education,} 746--746. 

\ind Brian Harvey, Daniel D Garcia, Tiffany Barnes, Nathaniel Titterton, Omoju Miller, Dan Armendariz, Jon McKinsey, Zachary Machardy, Eugene Lemon, Sean Morris, Josh Paley. 2014. ``\href{http://dl.acm.org/citation.cfm?id=2539022}{Snap!(Build your own Blocks).''} \emph{Proceedings of the 45th ACM Technical Symposium on Computer Science Education,} 749--749. 

\ind Omoju Miller, Jordan Arnesen, Eric Chu, Grayson Flood, Kevin Funkhouser, Alex Greenspan, Daniel Hsu. 2013. ``Hiphopathy: A Computational Thinking Process to Encourage Empathy.'' \emph{2013 California Cognitive Science Conference, Berkeley, CA.} 
 
%\end{revnumerate}
%\newpage
\bigskip
\noindent\emph{Essays and reviews \vspace{0.05in}}

%\renewcommand{\labelenumi}{\textsc{r}\theenumi.}
%\begin{revnumerate}


\ind Frieda McAlear, Omoju Miller, Dan Garcia, Alexis Martin. 2016. ``\href{http://www.csta.acm.org/Communications/sub/CSTAVoice_Files/csta_voice_01_2016.pdf}{Equity in Computer Science: From Exposure to Entrepreneurship.''} \emph{Computer Science Teacher's Association (CSTA) Voice, } 9--10. 

\ind Omoju Miller. 2014. ``\href{http://dl.acm.org/citation.cfm?id=2604994}{Its Deeper than Rap, Toward Culturally Responsive CS.'' }\emph{XRDS: Crossroads,  The ACM Magazine for Students, } 28--30. 

\ind Omoju Miller. 2013. ``\href{http://kaporcenter.org/wp-content/uploads/2013/10/Kapor_CodingLandscape_R3.pdf}{Winning with Both Sides of the Coin.''} \emph{Coding Nation Report} 6--7. 
% %\end{revnumerate}
 
 \bigskip

%\newpage

\marginhead{\sffamily {\vskip 0.5em}selected \newline conference \newline presentations \newline since 2012}
\medskip

\ind Keynote ``15+ Years of Learning Computational Intelligence.'' Annual OUSD Dinner with a Scientist, California, May 2016. 

\ind Invited panelist ``SPIE Women in Optics.'' SPIE Photonics West Annual Conference, California, February 2016. 

\ind Invited speaker ``Breaking Barriers: Changing the Landscape.'' Andela, Lagos, Nigeria, January 2016. 

\ind Invited panelist ``Behind the Click: Intersection of Women of Color in STEM and Social Justice.'' Black Girls Code, California, December 2015.

\ind Keynote ``Moonshot Thinking.'' Joint Board of Advisors Meeting, Purdue School of Engineering and Technology, IUPUI, Indiana, October 2015.

\ind Keynote ``Breaking Barriers: Changing the Landscape.'' Youth Start Up Weekend at Castlemont High School, California, May 2015. 

\ind Invited speaker, ``Breaking Barriers: Changing the Landscape.'' Office of Institutional Diversity, Harvey Mudd College, California, December 2014. 

\ind Keynote, ``\href{https://www.youtube.com/watch?v=owXez6sIRbY&ab_channel=IUCEWIT}{Inventing the Mythical Creature: Or How to Stay, Lead, and Delight in Tech as a Woman.}'' Annual Celebration of the IU Center of Excellence for Women in Technology (CEWiT), Indiana, October 2014.

\ind Invited presenter, ``Made With Code.'' Annual Dreamforce Developer's Conference, San Francisco, October 2014.

\ind Invited presenter, ``Inventing the Mythical Creature: Or How to Stay, Lead, and Delight in Tech as a Woman.'' Focus 100, New York, September 2014.

\ind Invited presenter, ``\#ThisIsGrit.'' White House Department of Education EdDataPalooza, Washington DC, January 2014.

\ind Invited panelist, ``Social Justice in the Age of Big Data.'' Swinging and Flowing: Inclusion and Diversity in the Age of Data. The Data and Democracy Initiative @ CITRIS Berkeley, Berkeley, April 2012.

\bigskip 

\marginhead{\sffamily {\vskip 0.6em}grants,\newline honors, \newline \& awards}
\medskip

\ind Levo 100: Transformer, Levo League, California, 2015.

\ind Bloomberg Beta Future Founder, Bloomberg Beta, California, 2014. 

\ind Winner, White House Dept of Edu EdDataPalooza, Washington DC, January 2014

\ind Kapor Capital Fellows, California, 2012.

\ind Invited participant, TEDxSummit, Doha, Qatar, 2011.

\ind Google CS Grad Forum Scholarship, California, 2010.

\ind SESAME Graduate Division Fellowship, UC Berkeley, 2010--2011

\ind Herff College of Engineering fellowship, Memphis, 2002--2006.

\bigskip 

\marginhead{\sffamily {\vskip 0.6em}press,\newline podcast, \newline \& coverage}
\medskip

\ind \href{http://best.berkeley.edu/2015/03/11/best-labber-in-berkeley-byte-qa-with-cs-ed-reformer-and-phd-candidate-omoju-miller/}{Q\&A with CS Ed Reformer and PhD Candidate Omoju Miller}, Berkeley Bytes, March 11, 2014.

\ind \href{http://www.npr.org/sections/alltechconsidered/2014/10/05/351851015/fortune-tellers-step-aside-big-data-looks-for-future-entrepreneurs}{Fortune-Tellers, Step Aside: Big Data Looks For Future Entrepreneurs}, NPR's  All Tech Considered, October 5, 2014.

\ind \href{https://www.blacksintechnology.net/the-road-to-50-podcast-african-americans-on-the-role-of-higher-education-in-innovation/}{African Americans on the role of Higher Education in Innovation}, \#BITTechTalk, December, 2014.

\ind \href{http://roadtripnation.com/leader/omoju-miller}{RoadTrip Nation: Omoju's Open Road}, Pivot TV, March, 2015.

\ind \href{http://hanselminutes.com/488/redesigning-computer-science-101-education-with-omoju-miller}{Redesigning Computer Science 101 Education with Omoju Miller}, The Hanselman Minutes Podcast: Fresh Air for Developers, August, 2015.

\ind \href{https://www.youtube.com/watch?v=T0wQRr4RuqM&ab_channel=HBCUtoStartup}{Learners Guild, Investing in You to Code,} HBCU to Startup Podcast, March, 2016.

\ind \href{http://audio.javascriptair.com/e/022-jsair-the-science-of-people-in-tech-with-kate-edwards-omoju-miller-and-steve-andrews/}{The Science of People in Tech}, JavaScript Air Podcast, May, 2016.

\ind \href{https://www.youtube.com/watch?v=q5mmE05e82I&ab_channel=ScienceWithTom}{Everybody's Got Questions (Yup)}, Featured in Science with Tom parody rap video, May 2016.


\end{document}